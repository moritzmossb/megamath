\section{Numbers and Sequences}\label{sec:numb_sequ}

Although we already introduced the concept of fields in \change{\momo}{add reference to chapter 2}, the number system
we use today, i.e. the real numbers $\R$ are typically introduced in lectures about real analysis. Hence we follow this
approach, as the concept of the reals encapsulates a good part of what we are trying to establish here. 

\subsection{Establishing a Number System}\label{ssec:est_numb_sys}

A number system is a set of numbers, typically infinite, which contains objects called numbers. The origin of numbers 
lies in the counting of objects. Based on our observations on fields, we want to apply the properties of a field to 
our number system. 

\begin{definition}{Axioms of \mathm{peano}}{peano}
    Let $s\colon\N\monmorph\N$ be the successor function:
    \begin{itemize}
        \item $1\in\N$
        \item $\forall n\in\N\colon \exists n'\in\N\colon s(n) = s'$
        \item $\not\exists n\in\N\colon s(n) = 1$
        \item $\forall m,n\in\N\colon s(m) = s(n) \implies m = n$
        \item $1\in X\wedge \forall x\in X\colon s(x)\in X\implies \N\subseteq X$
    \end{itemize}
\end{definition}

The successor function $s$ is injective and produces the natural numbers by mapping $n\in\N$ to $n+1$, since 
the successor of $n$ is $n+1$. These axioms lead directly to the method of Proof by Induction.

\begin{definition}{Induction}{induction}
    Let $\mathcal{P}(n)$ be a predicate where $n$ is element of some (ordered) index-family $I$, where 
    $\forall k\in I\colon\mathcal{P}(k)$. If 
    we can prove that $\forall n\in\N\colon\mathcal{P}(n) \implies \mathcal{P}(s(n))$, then 
    $\exists \sigma\colon J\isomorph\N$.
\end{definition}

Proofs by induction are very powerful, since we can simply check whether $\mathcal{P}(n)\implies\mathcal{P}(n+1)$
for $n\in J$. A very common example is the gaussian sum-formula:
\begin{theorem}{Gaussian Sum Formula}{gauss_sum}
    Let $n\in N$, then
    \begin{equation}
        \sum_{k= 1}^{n} k = \frac{n(n+1)}{2}\label{eq:gauss_sum}
    \end{equation}
\end{theorem}
\begin{proof}
    We first find $n_0\in\N\colon \mathcal{P}(n_0)$. Hence we test $n_0=1$:
    \begin{align*}
        &\sum_{k= 1}^{1} k = 1\\
        &\frac{2}{2} = 1
    \end{align*}
    This checks out, so we may proof $\mathcal{P}(n)\implies\mathcal{P}(n+1)$, i.e.
    \begin{align*}
        \sum_{k= 1}^{n}k = \frac{n(n+1)}{2} \implies \sum_{k= 1}^{n+1}k = \frac{(n+1)(n+2)}{2}
    \end{align*}
    We use the fact, that we can split up sums into their summands:
    \begin{align*}
        &\sum_{k= 1}^{n+1}k = n+1 + \sum_{k= 1}^{n}k = n+1 + \frac{n(n+1)}{2} = \frac{2n+2+n^2+n}{2}
        =\frac{(n+1)(n+2)}{2}
    \end{align*}
    Where the last equality follows by $(n+1)(n+2) = n^2+3n+2$.
\end{proof}
