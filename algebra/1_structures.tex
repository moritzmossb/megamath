\section{Algebraic Structures}\label{sec:alg_structures}

\subsection*{Motivation}\label{ssec:alg_motivation}

Although the following sections may seem like a quite boring sea of compact, unreadable equations and proofs, it forms
the essence of a modern understanding of most mathematics used today, since it formalizes the relations that sets form
with operations acting upon them. This leads to structures like groups, rings or fields, which allow for the 
generalization of various viewpoints\footnote{although the most general viewpoint would be in fact 
Category Theory} into an abstract form. 

Coming from High School, algebra may just sound like solving 
complicated equations, like $x^2 + 1 = 0$, which in fact forms the origin of the study.
\improvement{\momo}{add more and better motivation}

\subsection{Groups}\label{ssec:groups}

\begin{definition}{Magmata}{alg_struct}
    Let $X$ be a set and $\circ\colon X^2\subj X$. We call $(X,\circ)$ an algebraic structure, or magma, iff
    \begin{align}
        \forall x_1,x_2 \in X\colon x_1\circ x_2 \in X.\label{eq:def_alg_struct}
    \end{align}
\end{definition}

Algebraic structures are a very broad and basic structure, the fundamental idea of group theory and everything that 
is build upon it. A trivial simple example is $(\N,+)$, where $+$ denotes regular addition. Contrasting that, 
$(\N,-)$, with regular subtractions is not an algebraic structure, since $\exists x_1,x_2\in\N\colon x_1-x_2\not\in\N$,
i.e. $1-2\not\in\N$. Hence given any set $X$ paired with an operation $\circ\colon X^2\subj X$, one should prove that
$(X,\circ)$ forms an algebraic structure, in order to apply any general proofs or methods. If $(X,\circ)$ is 
an algebraic structure, then we call $X$ closed under $\circ$. There is no standardized notation for this 
circumstance, hence if this is the case, you might want to state closure under $\circ$ at the beginning of your 
work. In order to prove that $X$ is closed under $\circ$, one might negate \cref{eq:def_alg_struct}, yielding
\begin{align}
    \exists x_1,x_2\in X\colon x_1\circ x_2\not\in X.\label{eq:neg_def_alg_struct}
\end{align}
Hence, if proving $(X,\circ)$ is an algebraic structure, we find a pair $x_1,x_2\in X$, where $x_1\circ x_2\not\in X$,
i.e. $2-1$. Generally, when you need to prove such general properties, using a contradiction is the proper method, since
finding a counterexample is most of the times a lot easier than proving said property for all elements of $X$.

Given an algebraic structure $(X,\circ)$, one might find triplets $(x_1,x_2,x_2)\in X^3$, where 
$(x_1\circ x_2) \circ x_3 = x_1\circ (x_2\circ x_3)$.\improvement{\momo}{write more intro about associativity}

\begin{definition}{Associativity and Semi-Groups}{associativity}
    Let $(X,\circ)$ be an algebraic structure. We call $\circ$ associative, iff
    \begin{align}
        \forall x_1,x_2,x_3 \in X \colon (x_1\circ x_2) \circ x_3 = x_1 \circ (x_2 \circ x_3).\label{eq:def_associativity}
    \end{align}
    We then call $(X,\circ)$ a semi- (or half-) group.
\end{definition}

Semi-groups are the most basic structure, that are actually useful and have some\footnote{although very limited} use 
in e.g. linear algebra\footnote{like $(\F^{m\times n},\cdot)$ with the regular matrix multiplication}.

\begin{theorem}{Associativity of Set Operations}{set_op_assoc}
    Let $X$ be a set, then the following structures are semi-groups:
    \begin{equation}
        \begin{aligned}
            &(\mathcal{P}(X),\cap)\\
            &(\mathcal{P}(X),\cup)\\
            &(\mathcal{P}(X),\triangle)\\
        \end{aligned}\label{eq:assoc_set_ops}
    \end{equation}
    and $(\mathcal{P}(X),\setminus)$ is just algebraic.
\end{theorem}
\begin{proof}[Proof for \autoref{thm:set_op_assoc}]\label{proof:assoc_set_ops}
    We already proved in \change{\momo}{add reference for proof} that $\cap$, $\cup$ and $\triangle$ are associative,
    and $\setminus$ is not. Hence we only have to prove that 
    $\forall Y_1,Y_2\in\mathcal{P}(X)\colon Y_1\circ Y_2\in\mathcal{P}(X)$.
    \begin{align*}
        &Y_1\cap Y_2 = Y\qquad (\forall y\in Y\colon y\in Y_1\wedge y\in Y_2) \wedge (\forall y_1\in Y_1,y_2\in Y_2
        \colon y_1\in X\wedge y_2\in X)\implies \forall y\in Y\colon y\in X\\
        &Y_1\cup Y_2 = Y\qquad (\forall y\in Y\colon y\in Y_1\vee y\in Y_2)\wedge (\forall y_1\in Y_1,y_2\in Y_2
        \colon y_1\in X\wedge y_2\in X)\implies \forall y\in Y\colon y\in X\\
        &Y_1\triangle Y_2 = Y\qquad (\forall y\in Y\colon (y\in Y_2\setminus Y_1)\vee (y\in Y_1\setminus Y_2))
        \implies \forall y\in Y\colon y\in X\\
        &Y_1\setminus Y_2 = Y\qquad \forall y\in Y\colon y\in Y_1 \implies \forall y\in Y\colon y\in X\\
        &Y_2\setminus Y_1 = Y\qquad \forall y\in Y\colon y\in Y_2 \implies \forall y\in Y\colon y\in X\\
    \end{align*}
    i.e. $(\mathcal{P}(X),\cap)$, $(\mathcal{P}(X),\cup)$, $(\mathcal{P}(X),\triangle)$ and $(\mathcal{P}(X),\setminus)$
    are closed under their relative operation.
\end{proof}

Given $(\N,\cdot)$, where $\cdot$ is the common multiplication, we observe that there exists pairs $(a,b)\in\N^2$,
where $a\cdot b = b$, which is the case for $a=1$. We call $1$ the left-neutral element of $\cdot$. In contrast, 
$1$ is right-neutral if $b\cdot 1 = b$.

\begin{definition}{Neutral Element and Monoids}{monoids}
    Let $(X,\circ)$ be a semi-group. We call $e\in X$ neutral, if it is left- and right neutral, i.e.
    $\forall x\in X\colon e\circ x = x = x\circ e$. If $X$ has a neutral element $e$, we call 
    $(X,\circ)$ a monoid.
\end{definition}

\begin{theorem}{Uniqueness of Neutral Elements}{neutral_unique}
    Let $(X,\circ)$ be a monoid with neutral element $e$, then $e$ is unique.
\end{theorem}